\chapter{Sprachbefehle}
\label{chap:Sprachbefehle}

%8.1
\section{Sprache}
%8.2
\section{Sprache und Gesten}
%8.3
\section{Kinect for Windows SDK und Spracherkennung}

Die Spracherkennung erfolgt in diesem Fall durch die Sprachengine des Kinect for Windows SDKs.

Zur Verwendung von Sprachkommandos mit jnect 

\subsubsection{W\"orter und Phrasen}



\subsubsection{User Interfaces f\"ur Spracheingabe}
% visual clues for users
% show words that can be accepted or have been recognized
Zur besseren Nutzbarkeit von Sprachkommandos ist es sinnvoll dem Nutzer ein entsprechendes Feedback zu geben. Dies

%8.4
\section{Sprachkommandos f\"ur Roboter}

Da bisher keine n\"aheren Informationen zum Roboter, welcher im zweiten Teil der Arbeit angesteuert werden soll, vorliegen, ist es an dieser Stelle schwer 
konkrete Sprachbefehle zu bestimmen, mit denen der Roboter sp\"ater fernsgesteuert werden soll.
Die einzige konkrete Information die vorliegt ist, dass es sich um einen mobilen Roboter handeln wird. 
Jedoch l\"asst auch dies offen, ob der Roboter sich nur zweidimensional auf einer Ebene bewegen wird oder ob es sich m\"oglicherweise um eine Drohne 
handeln wird, welche ferngesteuert wird. Diese kann sich auch dreidimensional fortbewegen.
\par\smallskip 

\subsubsection{Grunds\"atzliche Sprachbefehle}

Da es sich um einen mobilen Roboter handlen wird, k\"onnen insofern schon einmal Annahmen getroffen werden. Auf jeden Fall wird dieser sich vorw\"arts 
und r\"uckwarts bewegen k\"onnen. Hieraus ergeben sich die ersten Befehle: ``Forward'' und ``Backward''. Der Roboter bewegt sich nun. Davon ausgehend 
ist der n\"achste Befehl das Anhalten, also ``Stop''. Hierauf beschr\"anken sich aber schon die M\"oglichkeiten, die man an dieser Stelle hat um 
Befehle zu definieren. Eine Drehung oder ein Abbiegen f\"uhrt hier schon zu Problemen. 

Doch zun\"achst noch zu grunds\"atzlichen Befehlen die abh\"angig von aktuellen Zustand des Roboters sind. Einer weitere M\"oglichkeit besteht 
darin die Bewegung des Roboters zu beschleunigen oder zu verlangsamen. Hieraus ergeben sich die Befehle ``Faster'' und ```Slower''.


\subsubsection{Richtungsver\"anderungen}

Wie bereits erw\"ahnt ergeben sich bei Richtungsver\"anderungen schon die ersten Probleme. Es stellt sich die Schwierigkeit, wie man einen 
Richtungswechsel definiert. Soll sich der Roboter um 90 Grad drehen oder eine bestimmte, gesprochene Gradzahl. Auch stellt sich die Frage wie 
der Roboter sich fortbewegt. F\"ahrt dieser auf Ketten so dreht er sich auf der Stelle, das ist relativ unproblematisch. F\"ahrt er jedoch 
auf R\"andern, oder l\"auft gar auf Beinen muss die Richtungsver\"anderung
