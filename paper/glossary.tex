%
% vorher in Konsole folgendes aufrufen: 
%	makeglossaries makeglossaries dokumentation.acn && makeglossaries dokumentation.glo
%
\newacronym{GEF}{GEF}{Graphical Editing Framework}

\newacronym{EMF}{EMF}{Eclipse Modeling Framework}

\newacronym{LWJGL}{LWJGL}{Lightweight Java Game Library}

\newacronym{RCP}{RCP}{Rich client platform}

\newacronym{GUI}{GUI}{Graphical User Interface}

\newacronym{GUID}{GUID}{Globally unique identifier}

\newglossaryentry{OpenGL}{name={Open Graphics Library},description={OpenGL (Open Graphics Library) ist eine Spezifikation für eine plattform- und programmiersprachenunabhängige Programmierschnittstelle zur Entwicklung von 2D- und 3D-Computergrafik. Der OpenGL-Standard beschreibt etwa 250 Befehle, die die Darstellung komplexer 3D-Szenen in Echtzeit erlauben}}

\newglossaryentry{SDK}{name={Software Development Kit},description={Ein Software Development Kit (SDK) ist eine Sammlung von Werkzeugen und Anwendungen, um eine Software zu erstellen, meist inklusive Dokumentation. Mit diesem ist es Softwareentwicklern möglich, eigene darauf basierende Anwendungen zu erstellen}}

\newglossaryentry{Skeletal Tracking System}{name={Skelett Tracking System},description={Unter Motion Capture, oder auch Skelett Tracking System, wörtlich Bewegungs-Erfassung, versteht man ein Tracking-Verfahren, das es ermöglicht, jede Art von Bewegungen so zu erfassen und in ein von Computern lesbares Format umzuwandeln, dass diese die Bewegungen analysieren, aufzeichnen, weiterverarbeiten und zur Steuerung von Anwendungen verwenden können}}

\newglossaryentry{Beamforming}{name={Mikrofonarray},description={Beamforming ist ein Verfahren zur Positionsbestimmung von Quellen in Wellenfeldern (z. B. Schallfeldern). Entsprechende Vorrichtungen werden auch akustische Kamera, Mikrofonarray oder akustische Antenne genannt}}

\newglossaryentry{Kinect}{name={Kinect},description={Kinect ist eine Plattform zur Steuerung der Videospielkonsole Xbox 360, die seit Anfang November 2010 verkauft wird}}

\newglossaryentry{Bewegungsdetektion}{name={Bewegungsdetektion},description={Unter Bewegungsdetektion versteht man Methoden des Maschinellen Sehens, die auf die Erkennung von Fremdbewegung im Erfassungsbereich eines optischen Detektors (also dem Blickfeld der Maschine) abzielen}}

\newglossaryentry{Datenhandschuh}{name={Datenhandschuh},description={Der Datenhandschuh ist ein Eingabegerät in Form eines Handschuhs. Durch Bewegungen der Hand und Finger erfolgt eine Orientierung im virtuellen Raum},plural={Datenhandschuhe}}

\newglossaryentry{MMS}{name={Mensch-Maschine-Schnittstelle},description={Die Benutzerschnittstelle (nach Gesellschaft für Informatik, Fachbereich Mensch-Computer-Interaktion auch Benutzungsschnittstelle) ist die Stelle oder Handlung, mit der ein Mensch mit einer Maschine in Kontakt tritt},plural={Mensch-Maschine-Schnittstellen}}

\newglossaryentry{MCI}{name={Mensch-Computer-Interaktion},description={Die Mensch-Computer-Interaktion (englisch \emph{Human-Computer Interaction, HCI}) als Teilgebiet der Informatik beschäftigt sich mit der benutzergerechten Gestaltung von interaktiven Systemen und ihren Mensch-Maschine-Schnittstellen}}