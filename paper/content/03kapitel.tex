\chapter{Stand der Technik}
\label{chap:Stand der Technik}

\section{Hardware}
% Beschreibung und Vorgaben der genutzten Technik
\subsection{Kinect}
% genaue Auflistung aller Bestandteile \ldots ben\"otigte Software, Eigenschaften
% Mikrofon, RGB-Kamera, Infrarot \ldots
Kinect ist ein Ger\"at zur Dedektion von Bewegungen. Entwickelt wurde es von PrimeSense als Hardware zur Steuerung der Videokonsole 
XBOX 360 von Microsoft Corp. . Nach der Ank\"undigung im M\"arz 2010 \footnote{Pressemitteilung, der Ver\"offentlichung. Siehe \href{https://www.microsoft.com/en-us/news/press/2010/mar10/03-31PrimeSensePR.aspx}{Link}. Microsoft.com.  Abgerufen Dezember 20, 2012}
war die Erweiterung mit dem Erscheinungsdatum 10. Novmeber 2010 in der Ausf\"uhrung \textit{Kinect for XBOX 360} in Europa erh\"altlich \footnote{Erscheinungsdatum der \textit{Kinect for XBOX 360}. Siehe \href{http://www.bbc.co.uk/newsbeat/10996389}{Link}. BBC UK. Abgerufen Dezember 20, 2012}.
Nach einem sehr erfolgreichem Verkaufsstart \footnote{\href{http://www.guinnessworldrecords.com/records-9000/fastest-selling-gaming-peripheral/}{\enquote{Am schnellsten verkauftes Perepherieger\"at f\"ur Spiele}}. Guinnessworldrecords.com. Abgerufen Dezemeber 22, 2012}
und anhaltender Nachfrage\footnote{Takahashi, Dean (Januar 9, 2012). \href{http://venturebeat.com/2012/01/09/xbox-360-surpassed-66m-sold-and-kinect-has-sold-18m-units/}{\enquote{Xbox 360 surpasses 66M sold and Kinect passes 18M units}}. venturebeat. Abgerufen Dezember 20, 2012}.
k\"undigte Microsoft am 9. Januar 2012 ein weiteres Modell der Kinect, die sogenannte \textit{Kinect for Windows} f\"ur den 1. Februar 2012 an \footnote{\href{https://blogs.msdn.com/b/kinectforwindows/archive/2012/01/09/kinect-for-windows-commercial-program-announced.aspx?Redirected=true}{\enquote{Ank\"undigung der \textit{Kinect for Windows}}}. blogs.msdn.com. Abgerufen Dezember 22, 2012}.

\subsection {Roboter - Ausblick}
% offene Schnittstelle, Arbeiten f\"ur 2. Studienarbeit \ldots
\section{Software}

\subsection{Microsoft Kinect SDK}
% N\"otig f\"ur Steuerung der Kinect
\subsection{Java}
% Entwicklungsumgebung \ldots

\subsection{GUI}
% Darstellung der Anwendung, der Gesteneingabe etc.

\section{Mustererkennung}
% kurzer Einblick in m\"ogliche Techniken/Techologien f\"ur Gestenerkennung \ldots