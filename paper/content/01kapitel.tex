\chapter{Einleitung}
\label{chap:Einleitung}
Gestik im Sinne von kommunikativen Bewegungen stellt einen wesentlichen Teil der nonverbalen Kommunikation dar. Die Definition nach Kurtenbach und Hulteen (1990) besagt:
\begin{quote}
A gesture is a motion of the body that contains information. Waving goodbye is a gesture.
Pressing a key on a keyboard is not a gesture because the motion of a finger on its way to hitting a key on a keyboard is neither observed nor significant.
All that matters is which key was pressed.
\end{quote}
Als solches bildet eine Geste abstrakte Strukturen und bildhafte Vorstellungen unmittelbar ab. Sie ist somit eine K\"orperbewegung, 
die Informationen enth\"alt. Informationen, die herk\"ommliche Eingabeger\"ate, darunter Tastatur und Maus, nicht in dieser Form wiedergeben k\"onnen.
Ein Umstand, der sich besonders im Bereich der \gls{MCI} auswirkt.
\newline
Das Ziel der \gls{MCI} ist es, die Kommunikation intuitiv und unmittelbar zu gestalten.
Genau an dieser Stelle setzen Gesten an, in dem sie eine bessere Schnittstelle zwischen Mensch und Maschine darstellen.
\newline
Die Spannweite von Realisierungen von \gls{MMS} ist gro\ss .
Fr\"uhe Ans\"atze zeigen bereits das Potential, Gesten zur Steuerung und Nutzung von diversen Ger\"aten zu verwenden.
Das \textit{Theremin} - 
ein 1919 erfundenes elektornisches Musikinstrument, das mit H\"anden ber\"uhrungslos durch Beeinflussung eines elektromagnetischen Feldes 
gesteuert wird und dadurch T\"one erzeugt - ist eine der ersten technischen L\"osungen f\"ur eine Gestensteuerung.
Eine weitere From, der \textit{\gls{Datenhandschuh}}, 1977 von Electronic Visualization Laboratory Labs entwickelt
\footnote{Sturman, D.J., Zeltzer, D. (January 1994). \enquote{A survey of glove-based input}. IEEE Computer Graphics and Applications 14 (1): 30–39}, 
erregte gro\ss es Interesse und wurde unter anderem als \textit{Power Glove} von Mattel vertrieben.
\glspl{Datenhandschuh} werden bis heute in den verschiedensten Bereichen eingesetzt.
\newline
Eine weitere Variante der Gestenerkennung ist die \gls{Bewegungsdetektion}. Dabei wird \"uber optische Sensoren der K\"orper oder einzelne K\"orperteile
des Nutzers erkannt und somit die Steuerung durch Gesten erm\"oglicht. Eine fr\"uhe L\"osung ist der sogenannte \textit{Videoplace} \footnote{\href{http://www.medienkunstnetz.de/works/videoplace/}{Beschreibung der Einrichtung \enquote{VIDEOPLACE}}. medienkunstnetz.de. Abgerufen Dezember 22, 2012},
 entwickelt von Myron Krueger Mitte der 70'er Jahre \footnote{Myron Krueger. Artificial Reality 2, Addison-Wesley Professional, 1991. ISBN 0-201-52260-8}.
Ausgelegt als ein \textit{Labor f\"ur k\"unstliche Realit\"at} war eine Person, durch ihn umgebende Projektoren und Videokameras, in der Lage seine Umgebung zu beeinflussen.
Moderne L\"osungen haben das Konzept der Detektion durch Kameratechnik weiterentwickelt und einer weiten \"Offentlichkeit zug\"anglich gemacht. Darunter auch das \textit{PLAYSTATIONEye}
\footnote{\href{http://us.playstation.com/ps3/accessories/playstation-eye-camera-ps3.html}{\enquote{PLAYSTATIONEye Brings Next-Generation Communication to PLAYSTATION3}}. us.playstation.com. Sony Computer Entertainment America. Abgerufen Dezember 21, 2012}.
Der popul\"arste Vertreter ist \textit{\glslink{Kinect}{Kinect for XBOX}} mit \"uber 18 Millionen verkauften Exemplaren
\footnote{Takahashi, Dean (Januar 9, 2012). \href{http://venturebeat.com/2012/01/09/xbox-360-surpassed-66m-sold-and-kinect-has-sold-18m-units/}{\enquote{Xbox 360 surpasses 66M sold and Kinect passes 18M units}}. venturebeat. Abgerufen Dezember 20, 2012}.
Im folgenden wird mit jener Kinect gearbeitet. 
\newline
In dieser Arbeit wird mit Hilfe der Kinect eine Gesten- und Spracherkennung entwickelt um eine Steuerung f\"ur einen mobilen Roboter zu implementieren.
Aufgrund des Umfangs dieser Studienarbeit wird die Erarbeitung dieser Anwendung in zwei Teile aufgeteilt.
\newline
Der erste Teil der Arbeit besch\"aftigt sich vorranging mit der Realisierung der Gesten- und Spracherkennung.
Die eigentliche Steuerung f\"ur einen mobilen Roboter wird vorerst durch eine Schnittstelle und einer Testumgebung ersetzt.
Der Fokus liegt hierbei in erster Linie auf der verwendeten Technik, in diesem Fall der Kinect zur \gls{Bewegungsdetektion}, den verwendeten Gesten und Sprachbefehlen,
sowie der Umsetzung der Software und den darin verwendeten Technologien.
Der zweite Teil wid an diese Arbeit ankn\"upfen und im weiteren die Umsetzung der Steuerung f\"ur und den Einsatz eines mobilen Roboters beschreiben.
\newline
Die folgende Ausarbeitung spiegelt die Umsetzung des ersten Teils dieser Arbeit wieder.