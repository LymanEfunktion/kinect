\chapter{Lego-Framework}
\label{chap:Lego-Framework}

% Auflistung
% Analyse 
% Auswahl: LeJos

\section{Programmierung}

Die Programmierung des NXT findet mittels der Entwicklungsumgebung NXT-G statt. Hierbei handelt es sich um einen grafischen Editor, mit dessen Hilfe die Logik des Roboters umgesetzt werden kann. Die Sensoren und Aktoren des Lego Roboters werden als Blöcke dargestellt, über die verschiedene Einstellungen vorgenommen werden können. Weiter Blöcke unterstützen simple Programmlogiken wie Schleifen und Bedingungen. 
\par\smallskip 
Mit Hilfe dieser Anwendung ist zwar eine leichte Programmierung des NXT m\"oglich, ist aber f\"ur dieses Projekt nicht sinnvoll einsetzbar. Die Firmware des NXT, sowie die Spezifikation der Sensoren und Aktoren, wurden von Lego ver\"offentlicht. Dar\"uber hinaus auch diverse Developer Toolkits. Durch diese Unterst\"utzung sind eine Vielzahl an Frameworks in verschieden Programmiersprachen und Umgebungen entstanden mit denen der NXT-Baustein programmiert oder auch aus der Ferne gesteuert werden kann.
\par\smallskip 
