\chapter{Sprachbefehle}
\label{chap:Sprachbefehle}

<<<<<<< HEAD
\section{Sprache}

Sprache ist f\"ur Menschen die nat\"urlichste Form der der Verst\"andigung. So definiert Edward Sapir (1921): 
\begin{quote}
Sprache ist eine ausschlie\ss lich dem Menschen eigene, nicht im Instinkt wurzelnde Methode zur \"ubermittlung von 
Gedanken, Gef\"uhlen und W\"unschen mittels eines Systems von frei geschaffenen Symbolen\footnotemark[1]
\end{quote}
Menschen nutzen Sprache zum Ausdruck ihrer selbst, sowie f\"ur Beschreibung und Manipulation ihrer Umgebung.

\footnotetext[1]{Zitiert nach John Lyons, 4.Auflage, 1992, S.13}

\section{Sprache und Gesten}

Da die menschliche Sprache sehr komplex und auch unpr\"azise ist, ist die die korrekte Deutung der erkannten Sprache f\"ur eine Maschine sehr komplex.
Stellt man jedoch den Bezug zu weiteren menschlichen Interaktionen her, wie die Gestik, so wird die Deutung der Bedeutung erheblich vereinfacht.
Umgekehrt kann durch Sprache beschrieben werden, was durch eine Geste ungenau ausgedr\"uckt wurde.
Cohen~\cite{bib:cohen1,bib:cohen2} zeigt hierzu, dass Sprache sich besser zur Beschreibung eignet als Gesten und das umgekehrt Gesten besser f\"ur
eine direkte Manipulation geeignet sind.
Dass eine Kombination dieser beiden Modalit\"aten f\"ur Nutzern bei grafischen Aufgaben bevorzugt werden zeigen auch Hauptman und McAvinney~\cite{bib:hauptmann}.
=======

\section{Sprache}

Sprache ist die nat\"urlichste Form der der Kommunikation zwischen Menschen. 

\section{Sprache und Gesten}

Verbindet man die beiden Modalit\"aten Sprache und Geste wird es m\"oglich Gedanken und Handlungsanweisungen einfacher auszudr\"ucken. Dies zeigt auch Cohen~\cite{bib:}
>>>>>>> 2bb7e10335176dc446d28fa0516733abbf1cc9ef

\section{Kinect for Windows SDK und Spracherkennung}

Die Spracherkennung erfolgt in diesem Fall durch die Sprachengine des Kinect for Windows SDK.

<<<<<<< HEAD
Zur Verwendung von Sprachbefehlen mit jnect wird das Interface \textit{SpeechListener} implementiert. Eine Implementierung ist in 
Listing~\ref{listing:speechlistener} zu sehen. Diese Implementierung wird dem \textit{KinectManager} \"ubergeben. Erkennt das 
Kinect~SDK eines der Worte, auf das es mittels der Methode \textit{getWords()} Zugriff hat, wird die Methode \textit{notifySpeech(...)} mit dem 
erkannten Wort als String aufgerufen. Umgesetzt ist der \textit{SpeechListener} derzeit im Action Bundle (Abschnitt~\ref{subsec:osgiaction}).

\par\smallskip
\lstset{language=Java}
\begin{lstlisting}[caption={jnect SpeechListener Implementation}, label={listing:speechlistener}]
public class SpeechlistenerImpl implements SpeechListener{
{
=======
Zur Verwendung von Sprachkommandos mit jnect wird das Interface SpeechListener implementiert. Eine Implementierung ist in Listing~\ref{listing:speechlistener} zu sehen. Diese Implementierung wird dem KinectManager 
\"ubergeben. Erkennt das KinectSDK eines der Worte, auf das es mittels der Methode \textit{getWords()} Zugriff hat, wird die Methode \textit{notifySpeech(...)} mit dem 
erkannten Wort als String aufgerufen.

\lstset{language=Java,
 basicstyle=\footnotesize, 
 numbers=left,
 captionpos=b,
 showspaces=false,             
 showstringspaces=false,}
\begin{lstlisting}[caption={jnect SpeechListener Implementation}, label={listing:speechlistener}]
SpeechListener helloWorldListener = new SpeechListener() {
>>>>>>> 2bb7e10335176dc446d28fa0516733abbf1cc9ef
			
	Set<String> words = new HashSet<String>();
			
	@Override
<<<<<<< HEAD
	public void notifySpeech(String speech) {
		System.out.println("Detected :" +speech);
	}
		
=======
		public void notifySpeech(String speech) {
		System.out.println("Detected :" +speech);
	}
			
>>>>>>> 2bb7e10335176dc446d28fa0516733abbf1cc9ef
	@Override
	public Set<String> getWords() {
		words.add("Hello");
		words.add("World");
		words.add("Hello World");
				
		return words;
	}
};
\end{lstlisting}
<<<<<<< HEAD
\par\smallskip
=======
>>>>>>> 2bb7e10335176dc446d28fa0516733abbf1cc9ef


\subsubsection{W\"orter und Phrasen}

<<<<<<< HEAD
Bei Auswahl der verwendbaren Befehle sollte neben ihrer Zweckm\"a\ss igkeit auch darauf geachtet werden, dass diese f\"ur die Sprachengine 
des Kinect SDK gut erkennbar bleiben~\cite{bib:kinect_hig}. Probleme k\"onnen hier komplexe oder exotische Worte bereiten. Aber auch einfache Worte k\"onnen schwer
erkennbar sein, wenn es viele \"ahnliche oder gleich klingende Worte in der Zielsprache gibt. Wichtig ist es hier die richtige Kombination aus
=======
Bei Auswahl der verwendbaren Befehle sollte neben ihrer Zweckm\"a\ss igkeit auch darauf geachtet werden, dass diese f\"ur die Sprach-Engine 
des Kinect SDK gut erkennbar bleiben~\cite{bib:kinect_hig}. Probleme k\"onnen hier komplexe oder exostische Worte bereiten. Aber auch einfache Worte k\"onnen schwer
erkennbar sein, wenn es viele \"ahnliche oder gleichklingende Worte in der Zielsprache gibt. Wichtig ist es hier die richtige Kombination aus
>>>>>>> 2bb7e10335176dc446d28fa0516733abbf1cc9ef
passenden und m\"ogichst klaren Worten zu finden. 

\subsubsection{User Interfaces f\"ur Spracheingabe}
% visual clues for users
% show words that can be accepted or have been recognized
Zur besseren Nutzbarkeit von Sprachkommandos ist es sinnvoll dem Nutzer ein entsprechendes Feedback zu geben~\cite{bib:kinect_hig}.
<<<<<<< HEAD
Sinnvoll hierf\"ur erweist es sich dem Nutzer visuell dar zu bieten, welchen aktuell nutzbaren Sprachbefehle verf\"ugbar sind.
=======
Sinnvoll hierf\"ur erweist es sich dem Nutzer visuell darzubieten, welchen aktuell nutzbaren Sprachbefehle verf\"ugbar sind.
>>>>>>> 2bb7e10335176dc446d28fa0516733abbf1cc9ef
Als ebenfalls hilfreich erweist sich die Ausgabe des erkannten Befehls, und eventuellen Nachfragen oder Verbesserungen, falls 
ein Befehl nicht eindeutig erkannt werden konnte.

%8.4
\section{Sprachkommandos f\"ur Roboter}

Da bisher keine n\"aheren Informationen zum Roboter, welcher im zweiten Teil der Arbeit angesteuert werden soll, vorliegen, ist es an dieser Stelle schwer 
<<<<<<< HEAD
konkrete Sprachbefehle zu bestimmen, mit denen der Roboter sp\"ater ferngesteuert werden soll.
=======
konkrete Sprachbefehle zu bestimmen, mit denen der Roboter sp\"ater fernsgesteuert werden soll.
>>>>>>> 2bb7e10335176dc446d28fa0516733abbf1cc9ef
Die einzige konkrete Information die vorliegt ist, dass es sich um einen mobilen Roboter handeln wird. 
Jedoch l\"asst auch dies offen, ob der Roboter sich nur zweidimensional auf einer Ebene bewegen wird oder ob es sich m\"oglicherweise um eine Drohne 
handeln wird, welche ferngesteuert wird. Diese kann sich auch dreidimensional fortbewegen.
\par\smallskip 

\subsubsection{Bewegung}

<<<<<<< HEAD
Da es sich um einen mobilen Roboter handelt wird, k\"onnen insofern schon einmal Annahmen getroffen werden. Auf jeden Fall wird dieser sich vorw\"arts 
=======
Da es sich um einen mobilen Roboter handlen wird, k\"onnen insofern schon einmal Annahmen getroffen werden. Auf jeden Fall wird dieser sich vorw\"arts 
>>>>>>> 2bb7e10335176dc446d28fa0516733abbf1cc9ef
und r\"uckwarts bewegen k\"onnen. Hieraus ergeben sich die ersten Befehle: ``Forward'' und ``Backward''. Der Roboter bewegt sich nun. Davon ausgehend 
ist der n\"achste Befehl das Anhalten, also ``Stop''. Hierauf beschr\"anken sich aber schon die M\"oglichkeiten, die man an dieser Stelle hat um 
Befehle zu definieren. Eine Drehung oder ein Abbiegen f\"uhrt hier schon zu Problemen. 

Doch zun\"achst noch zu grunds\"atzlichen Befehlen die abh\"angig von aktuellen Zustand des Roboters sind. Einer weitere M\"oglichkeit besteht 
darin die Bewegung des Roboters zu beschleunigen oder zu verlangsamen. Hieraus ergeben sich die Befehle ``Faster'' und ```Slower''.


\subsubsection{Richtungsver\"anderungen}

Wie bereits erw\"ahnt ergeben sich bei Richtungsver\"anderungen schon die ersten Probleme. Es stellt sich die Schwierigkeit, wie man einen 
Richtungswechsel definiert. Soll sich der Roboter um 90 Grad drehen oder eine bestimmte, gesprochene Gradzahl. Auch stellt sich die Frage wie 
der Roboter sich fortbewegt. F\"ahrt dieser auf Ketten so dreht er sich auf der Stelle, das ist relativ unproblematisch. F\"ahrt er jedoch 
<<<<<<< HEAD
auf R\"andern, oder l\"auft gar auf Beinen muss die Richtungsver\"anderung detaillierter beschrieben werden. 
=======
auf R\"andern, oder l\"auft gar auf Beinen muss die Richtungsver\"anderung detailierter beschrieben werden. 
>>>>>>> 2bb7e10335176dc446d28fa0516733abbf1cc9ef

\subsubsection{Geste und Sprache in Kombination}

Bei dem zuvor beschriebenen Problem bietet es sich an Sprache und Geste zu kombinieren. W\"ahrend es durch Sprache schwierig ist, den Winkel in 
welchem die Richtungs\"anderung stattfinden soll zu benennen, so wird dies erheblich vereinfacht durch eine weisende Geste, wie in~\ref{subsec:gesture-extforward} beschrieben.

Je nach Art des Roboters kann es auch noch weitere sinnvolle Kombinationen geben. Da dies an dieser Stelle jedoch zu weit in den Bereich der Spekulation geht,
wird sich dieser Problematik im zweiten Teil dieser Arbeit gewidmet.
<<<<<<< HEAD
=======




% Cohen, P.R. (1992) The Role of Natural Language in a Multimodal Interface. Proceedings of the 
%  7th Annual ACM Symposium on User Interface Technology (UIST ’92),143-149.

% Cohen, P.R., Dalrymple, M., Pereira, F.C.N., Sullivan, J.W., Gargan Jr., R.A., Schlossberg, J.L. 
%  and Tyler, S.W. (1989) Synergistic Use of Direct Manipulation and Natural Language. 
%  Conference on Human Factors in Computing Systems (CHI '89), pp. 227-233. Austin, Texas, 
%  IEEE, ACM, 1989.
>>>>>>> 2bb7e10335176dc446d28fa0516733abbf1cc9ef
