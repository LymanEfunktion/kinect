\chapter{Aufgabenstellung}
\label{chap:Aufgabenstellung}

\section{Problemstellung und Ziel der Arbeit}

% Auszug Anmeldung
% Erstellung einer Steuerung auf Basis von Gesten und Sprache für einen mobilen Roboter
% Verwendung der Kinect for Windows und der Kinect SDK zur Erkennung der Gesten und Sprache
% Entwicklung der Anwendung in der Programmiersprache Java
% Untersuchung des Frameworks OpenNI

% Notizen
% Schnittstelle von einem Human Interface Device, in diesem Fall die Microsoft Kinect zu einem mobilen Roboter
% Entwicklung unter Nutzung des Microsoft Kinect SDK
% Entwicklung mit Java 
% Untersuchung des Frameworks OpenNI und Evaluation anderer Optionen
%

%
Das Ziel des ersten Teils dieser Arbeit ist die Entwicklung einer Schnittstelle zwischen Mensch und Roboter.
% Hierzu soll der Mensch mittels Sprache und Gesten die Möglichkeit der Fernsteuerung eines mobilen Roboters erlangen. 
% Mittels Gesten und Sprache soll die Steuerung eines mobilen Roboters ermöglicht werden.
Über dieser Schnittstelle soll es dem Benutzer m\"oglich sein, mittels Sprache und Gesten einen mobilen Roboter fernzusteuern. Die Entwicklung der Ansteuerung des mobilen Roboters wird im zweiten Teil der Arbeit erl\"autert.
\newline
Zur Erfassung von Sprache und Bewegung des Benutzers soll im Rahmen dieser Arbeit die \textit{\glslink{Kinect}{Kinect}} von Microsoft und das zugehörige \textit{\glslink{SDK}{SDK}} \footnote{Microsoft Corporation (2012). \href{http://www.microsoft.com/en-us/kinectforwindows/} Abgerufen Januar 05, 2013} eingesetzt werden. 
Die Entwicklung dieser Anwendung soll auf Basis der Programmiersprache Java erfolgen.
% TODO VW: Erneutes verlinken SDK? TODO VW: C++, C# und VB ins Glossary?
Hierzu ist es erforderlich eine Schnittstelle zum Kinect SDK zu verwenden, da hier native Entwicklung nur mit C++, C\# oder Visual Basic möglich ist.
% TODO VW: gls Framework gls SDK
Daher muss an dieser Stelle ein Framework eingesetzt werden, das den Zugriff auf das SDK mittels Java ermöglicht.
% TODO VW: gls OpenNI
In diesem Rahmen soll das Framework OpenNI und etwaige Alternativen evaluiert werden.


\section{Geplantes Vorgehen}

% Auszug Anmeldung
% Machbarkeitsvorstellung einer Steuerung durch genannte Kinect
% Selektion eines passenden Modells eines mobilen Roboters zur Verwendung mit der Anwendung
% Erstellung einer Anwendung zur Steuerung eines solchen Roboters
% Weitere Tests und Anwendungsszenarien für die Steuerung mittels Kinect

% Notizen
% Erarbeitung der theoretischen Grundlagen (Kinect, Gesten, Sprache, Roboter)
% Erarbeitung der Umsetzungsmöglichkeiten (Kinect SDK, jNect, OpenKinect, OpenNI)
% Entwicklung einer Anwendung auf Basis der Erkenntnisse 
% Evaluation weiterer Test und Anwendungsszenarien zu Steuerung durch die Kinect
%

Es müssen die Grundlagen zur Entwicklung der zuvor beschriebenen Anwendung geschaffen werden. 
Hierzu ist es zun\"achst erforderlich die notwendigen theoretischen Grundlagen zur Sprach- und Gestensteuerung, sowie deren Erkennung und Verarbeitung zu erarbeiten.
% gls Kinect SDK?
Ebenso müssen  zur Entwicklung mit dem Kinect SDK in Kombination mit der Programmiersprache Java m\"oglichen Optionen ermittelt und evaluiert werden.
Auf Basis der hierdurch gewonnenen Erkenntnisse soll im ersten Schritt eine Anwendung entwickelt werden, die es erm\"oglicht durch Sprache und Gesten definierte Aktionen zu erkennen. 
Diese Aktionen werden die Grundlage der Ansteuerung des mobilen Roboters darstellen.
\newline
Nach M\"oglichkeit sollen dar\"uber hinaus weitere Anwendungsszenarien zur Steuerung durch die Kinect erdacht und evaluiert werden.

\section{Ausblick}

% Vorzeitiger Ausblick
% Erarbeitung der Grundlagen zur Steuerung eines bisher nicht bekannten mobilen Roboters
% Erweiterung des entwicklten Frameworks zur Ansteuerung des Roboters
% Erweiterung des Gesten- und Sprachrepertoires

Im zweiten Teil dieser Arbeit soll die Schnittstelle von der bisher entwickelten Sprach- und Gestenerkennung zu einem bisher nicht näher spezifizierten mobilen Roboter geschaffen werden.
Hierzu wird es notwendig die enstprechenden Grundlagen zu erwerben und die Anwendung um die Implementierung etwaiger Roboter-Aktionen zu erweitern.
Dar\"uber hinaus soll das Repertoire an zu erkennenden Aktionen um im Zusammenhang einer Roboter-Steuerung sinnvollen Kommandos erweitert werden.