\chapter{Auswahl des Roboters}
\label{chap:Roboterauswahl}

% Definition - mobile Roboter
% Beschreibung: warum mobiler Roboter
% Analyse möglicher Typen - Verfügbarkeit - Features - Nutzen - Umsetzbarkeit
% Auswahl: Begründung für Lego

\section{Mobile Roboter}


Diese Arbeit besch\"aftigt sich mit der Steuerung eines Roboters aus der Ferne. Als Steuerungsinterface wird in diesem Fall jedoch kein traditionelles Interface, wie beispielsweise ein Joystick, verwendet, sondern es kommt das, im ersten Teil dieser Studienarbeit vorgestellte Interface, dass auf Sprach- und Gestensteuerung basiert zum Einsatz. 

\section{M\"ogliche Robotermodelle}

\section{LEGO Mindstorms NXT}

Im Rahmen dieses Projektes wird der NXT 2.0 aus der Lego Mindstorm Serie eingesetzt. Im Kontext dieser Arbeit sind keine Sensoren notwendig, da der Roboter aus der Ferne gesteuert wird. 