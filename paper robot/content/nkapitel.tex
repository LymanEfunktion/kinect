\chapter{Ausblick - weitere Arbeiten}
\label{chap:Ausblick}

Im Rahmen dieser Arbeit wurde ein Anwendung und ein Framework zur Steuerung eines Roboters mittels Gesten und Sprachbefehlen entwickelt. Die dazu erforderlichen theoretischen Grundlagen wurden in einer ersten Arbeit \cite{bib:Ebner_Werling}, welche sich haupts\"achlich mit der Umsetzung des User-Interface mit Gesten und Spracheingabe, erarbeitet und mit den Erkenntnissen dieser Arbeit, die sich mit der Umsetzung der Schnittstelle zu einem mobilen Roboter widmet, verkn\"upft. In der ersten Arbeit wurden ausf\"uhrlich die verschiedenen Verfahren zur Erkennung von Gesten und Sprache erl\"autert, sowie im speziellen der verwendete Ansatz derversteckten Markov Modelle. In dieser Arbeit sind vor allem die Grundlagen der Entwicklung mit einem Roboter, in diesem Fall der Lego Mindstorms NXT dargelegt. Die entwickelte Softwarel\"osung soll der Dualen Hochschule Karlsruhe zur weiteren Verwendung zur Verf\"ugung gestellt werden. 
\par\smallskip
Wie im allgemeinen \"ublich ist eine Software niemals wirklich fertig gestellt, es wird immer noch Potential zur Verbesserung oder Weiterentwicklung geben. Das entwickelte Framework stellt eine gute Grundlage zur weiteren Verwendung der Kinect von Microsoft zur Entwicklung einer m\"achtigen Benutzerschnittstelle zur nat\"urlichen Steuerung von Robotern. Die besondere Herausforderung lag hier vor allem in der Umsetzung der \glslink{HMM}{Hidden Markov Modells} zur Erkennung von Gestern anhand von Koordinaten.
\par\smallskip
Noch notwendige Anpassungen und Verbesserungen werden von den Durchf\"uhrenden, Herrn Ebner und Herrn Werling, soweit m\"oglich auch nach Abschluss der Arbeit noch durchgef\"uhrt werden. Hierzu z\"ahlen unter anderem die Umsetung einer Graphischen Oberfl\"ache zur besseren Verwendung des Roboters, sowie das entfernen m\"oglicher Fehlerquellen in den bereits entwickelten Komponenten und auch das anfertigen von besserer Dokumentation zu Verwendung.