\pagestyle{empty}

\begin{abstract}
Verwendet man heutzutage einen Computer oder eine Spielekonsole, so benutzt man in den meisten F\"allen noch immer einen Controller, Maus und Tastatur, oder irgendein anderes Eingabeger\"at. Doch immer h\"aufiger werden moderne Formen, wie die \gls{Bewegungsdetektion} verwendet. Dabei wird mittels Videokamera der Benutzer erkannt und dieser ist in der Lage, mit vermindertem Einsatz von herk\"ommlichen Eingabeger\"aten oder ganz ohne diese eine Anwendung, oder gar Spiele zu steuern. Kinect for Windows ist ein popul\"arer Vertreter dieses modern anma\ss enden, aber in Wahrheit doch recht alten Verfahrens der Eingabesteuerung. 
\newline
In den folgenden Seiten wird erl\"autert, wie mit einer Kinect for Windows eine Steuerung f\"ur einen mobilen Roboter entworfen werden, und aussehen kann. Dabei werden Techniken und Modelle f\"ur die Gestenerkennung, ebenso Sprachbefehle und die Umsetzung einer solchen Anwendung erl\"autert.
\newline
Die Autoren Ebner und Werling haben diese Arbeit in Zusammenarbeit verfasst, wobei die Kapitel jeweils von einem der Autoren geschrieben wurden. Auf Herrn Ebner entfallen Kapitel~\ref{chap:Einleitung},~\ref{chap:Stand der Technik},~\ref{chap:Konzeption},~\ref{chap:Modelle} und~\ref{chap:Gesten} und auf Herrn Werling die Kapitel~\ref{chap:Aufgabenstellung},~\ref{chap:Kinect},~\ref{chap:Sprachbefehle},~\ref{chap:Implementierung} und~\ref{chap:Ausblick}.
\end{abstract}