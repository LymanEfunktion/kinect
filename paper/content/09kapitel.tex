\chapter{Implementierung}
\label{chap:Implementierung}
% Blablabla

%M\"ogliche Punkte: 

%Architektur\"ubersicht/OSGi-Service
%Kinect Bundle
%Gesten Bundle(s)
%Action Bundle(s)
%RobotAction Bundle(s)
%User Interface

%Architektur\"ubersicht/OSGi-Service
%spricht f\"ur sich

%Kinect Bundle
%Kinect lauscht ob Gesture(s), GestureListener und SpeechListener als Services verf\"ugbar sind, 
%sind diese verf\"ugbar werden sie eingebunden

%Gesten Bundle
%Geste wird in einem Bundle implementiert und dort als Service ver\"offentlicht
%Geste wird von Kinect gefunden und dem Repertoire hinzugef\"ugt

%Action Bundle
%Eine Action sucht seine Geste, und kann erst starten, wenn es diese gefunden hat
%Action implementier GestureSpeechListener und SpeechListener
%Action published sich als Service, wird von Kinect gefunden, diese added SpeechListener und GestureListener
%Bei Kombination von beidem muss die Action entscheiden, was sie tun will (zeitliche Abfolge usw.)

%RobotAction

%bisher einfach eine Klasse, sollte das auch als Service raus und von Action gesucht werden?

%UserInterface

%quasi nicht existent
%Evtl anzeigen des GEF Bildes, Einblendungen, welche Sprachbefehle und Gesten erkannt wurden

\section{OSGI--Bundles}
% Aufbau der OSGI Archiktur der Anwendung erl\"autern, evtl. Grafik \ldots UML

\subsection{Gesten}
% Beschreibung des Bundles
\subsection{Sprachbefehle}
% Beschreibung des Bundles
\subsection{Roboterschnittstelle}
\label{subsec:Roboterschnittstelle}
% Beschreibung des Bundles 
\subsection{Ausf\"uhrbare Aktionen}
% Beschreibung des Bundles

\section{Oberfl\"ache}
% Wie benutzt der Nutzer die Anwendung, was kann er sehen \ldots
\subsection{RCP--Anwendung}
% Warum RCP \ldots
\subsection{Grafische Darstellung}
% Darstellung der Geste etc.

\section{Ablaufbeschreibung}
% Erl\"auternung der Anwendung an Beispiel einer Geste
