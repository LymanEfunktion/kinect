%
% vorher in Konsole folgendes aufrufen: 
%	makeglossaries makeglossaries dokumentation.acn && makeglossaries dokumentation.glo
%
\newacronym{HMM}{HMM}{Hidden Markov Model}

\newacronym{IDE}{IDE}{Integrated development environment}

\newacronym{GEF}{GEF}{Graphical Editing Framework}

\newacronym{EMF}{EMF}{Eclipse Modeling Framework}

\newacronym{LWJGL}{LWJGL}{Lightweight Java Game Library}

\newacronym{RCP}{RCP}{Rich client platform}

\newacronym{GUI}{GUI}{Graphical User Interface}

\newacronym{GUID}{GUID}{Globally unique identifier}

\newacronym{DTW}{DTW}{Dynamic time warping}

\newacronym{NUI}{NUI}{Natural User Interface}

\newacronym{OUI}{OUI}{Organic user interface}

\newacronym{RIA}{RIA}{Robotic Industries Association}

\newacronym{JARA}{JARA}{Japan Robot Association}

\newglossaryentry{Panto}{name={Pantomime},description={Pantomime bezeichnet sowohl eine Form der darstellenden Kunst, deren Darsteller in den meisten F\"allen ohne gesprochenes Wort auskommen, als auch den K\"unstler selbst, der diese Form der Darstellung praktiziert}}

\newglossaryentry{Ikoni}{name={Ikonizit\"at},description={Ikonizit\"at ist ein mehrdeutiger linguistischer Fachbegriff, der sich auf den Begriff des Ikons im Sinne von Charles Sanders Peirce bezieht. Er bezeichnet unter anderem, die Beziehung zwischen dem Ausdruck und Inhalt ikonischer Zeichen. Ikonizit\"at wird oftmals mit Abbildungsverh\"altnis sprachlicher Ausdr\"ucke \"ubersetzt}}

\newglossaryentry{Deixis}{name={Deixis},description={Deixis, auch indexikalische Semantik, ist ein Fachbegriff aus der Sprachwissenschaft. Er bezeichnet die Bezugnahme auf Personen, Orte und Zeiten im Kontext, die mit Hilfe von deiktischen oder indexikalischen Ausdr\"ucken wie \textit{ich, du, dort, hier, morgen, heute ... } erfolgt}}

\newglossaryentry{Epistemik}{name={Epistemik},description={Die epistemische Logik, auch Wissenslogik befasst sich mit Glauben und Wissen bei Individuen sowie Gruppen. Ziel von Untersuchungen mittels epistemischer Logik ist oft ein dynamisches oder flexibles Modell von Meinungs- und Wissenszust\"anden}}

\newglossaryentry{Semiotik}{name={Semiotik},description={Semiotik ist die Wissenschaft, die sich mit Zeichensystemen aller Art (zum Beispiel: Bilderschrift, Gestik, Formeln, Sprache, Verkehrszeichen) befasst. Sie ist die allgemeine Theorie vom Wesen, der Entstehung (Semiose) und dem Gebrauch von Zeichen}}

\newglossaryentry{Ergo}{name={Ergodizit\"at},description={Ergodizit\"at ist eine Eigenschaft dynamischer Systeme. Sie bezieht sich auf das mittlere Verhalten eines Systems. Ein solches System wird durch eine Musterfunktion beschrieben, die die zeitliche Entwicklung des Systems abh\"angig von seinem aktuellen Zustand bestimmt}}

\newglossaryentry{DP}{name={Dynamische Programmierung},description={Dynamische Programmierung ist eine Methode zum algorithmischen L\"osen von Optimierungsproblemen. Der Begriff wurde in den 1940er Jahren von dem amerikanischen Mathematiker Richard Bellman eingef\"uhrt, der diese Methode auf dem Gebiet der Regelungstheorie anwandte. In diesem Zusammenhang wird auch oft von Bellmans Prinzip der dynamischen Programmierung gesprochen}}

\newglossaryentry{KNN}{name={K\"unstliche neuronale Netze},description={K\"unstliche neuronale Netze basieren meist auf der Vernetzung vieler McCulloch-Pitts-Neuronen oder leichter Abwandlungen davon. Grunds\"atzlich k\"onnen auch andere k\"unstliche Neuronen Anwendung in KNNen finden, z.B. das High-Order-Neuron. Die Topologie eines Netzes (die Zuordnung von Verbindungen zu Knoten) muss abh\"angig von seiner Aufgabe gut durchdacht sein. Nach der Konstruktion eines Netzes folgt die Trainingsphase, in der das Netz „lernt“}}

\newglossaryentry{StochProz}{name={Stochastischer Prozess},description={Ein stochastischer Prozess ist die mathematische Beschreibung von zeitlich geordneten, zuf\"alligen Vorg\"angen}}

\newglossaryentry{NeuroNetz}{name={K\"unstliches neuronales Netz},description={K\"unstliche neuronale Netze (selten auch k\"unstliche neuronale Netzwerke, kurz: KNN, engl. artificial neural network – ANN) sind Netze aus k\"unstlichen Neuronen. Sie sind ein Zweig der k\"unstlichen Intelligenz und prinzipieller Forschungsgegenstand der Neuroinformatik}, plural={k\"unstliche neuronale Netze}}

\newglossaryentry{Equinox}{name={Equinox},description={Equinox (von englisch Tag- und Nachtgleiche) ist ein von der Eclipse Foundation entwickeltes Java-basiertes Framework, welches die OSGi-Kernspezifikation implementiert und das Ger\"ust der integrierten Entwicklungsumgebung Eclipse bildet}}

\newglossaryentry{API}{name={Programmierschnittstelle},description={Eine Programmierschnittstelle (englisch application programming interface, API; deutsch Schnittstelle zur Anwendungsprogrammierung) ist ein Programmteil, der von einem Softwaresystem anderen Programmen zur Anbindung an das System zur Verf\"ugung gestellt wird, plural={API}}}

\newglossaryentry{Framework}{name={Framework},description={Ein Framework (englisch f\"ur Rahmenstruktur) ist ein Programmierger\"ust, das in der Softwaretechnik, insbesondere im Rahmen der objektorientierten Softwareentwicklung sowie bei komponentenbasierten Entwicklungsans\"atzen, verwendet wird}}

\newglossaryentry{Eclipse}{name={Eclipse},description={Eclipse (von englisch eclipse ‚Sonnenfinsternis‘, ‚Finsternis‘, ‚Verdunkelung‘) ist ein quelloffenes Programmierwerkzeug zur Entwicklung von Software verschiedenster Art. Urspr\"unglich wurde Eclipse als integrierte Entwicklungsumgebung (IDE) f\"ur die Programmiersprache Java genutzt, aber mittlerweile wird es wegen seiner Erweiterbarkeit auch f\"ur viele andere Entwicklungsaufgaben eingesetzt. F\"ur Eclipse gibt es eine Vielzahl sowohl quelloffener als auch kommerzieller Erweiterungen}}

\newglossaryentry{OpenGL}{name={Open Graphics Library},description={OpenGL (Open Graphics Library) ist eine Spezifikation f\"ur eine plattform- und programmiersprachenunabh\"angige Programmierschnittstelle zur Entwicklung von 2D- und 3D-Computergrafik. Der OpenGL-Standard beschreibt etwa 250 Befehle, die die Darstellung komplexer 3D-Szenen in Echtzeit erlauben}}

\newglossaryentry{SDK}{name={Software Development Kit},description={Ein Software Development Kit (SDK) ist eine Sammlung von Werkzeugen und Anwendungen, um eine Software zu erstellen, meist inklusive Dokumentation. Mit diesem ist es Softwareentwicklern m\"oglich, eigene darauf basierende Anwendungen zu erstellen}}

\newglossaryentry{Skeletal Tracking System}{name={Skelett Tracking System},description={Unter Motion Capture, oder auch Skelett Tracking System, w\"ortlich Bewegungs-Erfassung, versteht man ein Tracking-Verfahren, das es erm\"oglicht, jede Art von Bewegungen so zu erfassen und in ein von Computern lesbares Format umzuwandeln, dass diese die Bewegungen analysieren, aufzeichnen, weiterverarbeiten und zur Steuerung von Anwendungen verwenden k\"onnen}}

\newglossaryentry{Beamforming}{name={Mikrofonarray},description={Beamforming ist ein Verfahren zur Positionsbestimmung von Quellen in Wellenfeldern (z. B. Schallfeldern). Entsprechende Vorrichtungen werden auch akustische Kamera, Mikrofonarray oder akustische Antenne genannt}}

\newglossaryentry{Kinect}{name={Kinect},description={Kinect ist eine Plattform zur Steuerung der Videospielkonsole Xbox 360, die seit Anfang November 2010 verkauft wird}}

\newglossaryentry{Bewegungsdetektion}{name={Bewegungsdetektion},description={Unter Bewegungsdetektion versteht man Methoden des Maschinellen Sehens, die auf die Erkennung von Fremdbewegung im Erfassungsbereich eines optischen Detektors (also dem Blickfeld der Maschine) abzielen}}

\newglossaryentry{Datenhandschuh}{name={Datenhandschuh},description={Der Datenhandschuh ist ein Eingabeger\"at in Form eines Handschuhs. Durch Bewegungen der Hand und Finger erfolgt eine Orientierung im virtuellen Raum},plural={Datenhandschuhe}}

\newglossaryentry{MMS}{name={Mensch-Maschine-Schnittstelle},description={Die Benutzerschnittstelle (nach Gesellschaft f\"ur Informatik, Fachbereich Mensch-Computer-Interaktion auch Benutzungsschnittstelle) ist die Stelle oder Handlung, mit der ein Mensch mit einer Maschine in Kontakt tritt},plural={Mensch-Maschine-Schnittstellen}}

\newglossaryentry{MCI}{name={Mensch-Computer-Interaktion},description={Die Mensch-Computer-Interaktion (englisch \emph{Human-Computer Interaction, HCI}) als Teilgebiet der Informatik besch\"aftigt sich mit der benutzergerechten Gestaltung von interaktiven Systemen und ihren Mensch-Maschine-Schnittstellen}}

\newglossaryentry{OpenNI}{name={OpenNI},description={OpenNI oder Open Natural Interaction ist eine von der Industrie gef\"uhrte gemeinn\"utzige Organisation deren Ziel die Zertifizierung und Verbesserung von nat\"urlichen \glslink{NUI} und organischen Benutzeroberfl\"achen \glslink{OUI}, zugeh\"origen Ger\"aten und deren Anwendugen, sowie auch Anwendungen zum Zugriff auf solche Ger\"ate, ist}}

\newglossaryentry{Holonom}{name={Holonom},description={Ein holonomes System von K\"orpern zeichnet sich dadurch aus, dass sich die Lage der K\"orper durch n generalisierte Koordinaten q_1, q_2, \ldots , q_n beschreiben l\"asst, die g\"anzlich unabh\"angig voneinander sind}}

\newglossaryentry{MCS}{name={Man-Computer Symbiosis},description={Man-computer symbiosis is an expected development in cooperative interaction between men and electronic computers. It will involve very close coupling between the human and the electronic members of the partnership}}

\newglossaryentry{Gamification}{name={Gamification},description={Als Gamification oder Gamifizierung (seltener auch Spielifizierung) bezeichnet man die Anwendung spieltypischer Elemente und Prozesse in spielfremdem Kontext}}

\newglossaryentry{OSGi}{name={OSGi},description={Die OSGi Alliance (fr\"uher Open Services Gateway initiative) spezifiziert eine hardwareunabh\"\"angige dynamische Softwareplattform, die es erleichtert, Anwendungen und ihre Dienste per Komponentenmodell („Bundle“/„Service“) zu modularisieren und zu verwalten („Service Registry“). Die OSGi-Plattform setzt eine Java Virtual Machine (JVM) voraus und bietet darauf aufbauend das OSGi-Framework}}

\newglossaryentry{NXT}{name={Lego Mindstorms NXT},description={Der Lego Mindstorms NXT ist ein Roboter Modell des Spielwarenherstellers LEGO.}}