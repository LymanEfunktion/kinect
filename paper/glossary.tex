%
% vorher in Konsole folgendes aufrufen: 
%	makeglossaries makeglossaries dokumentation.acn && makeglossaries dokumentation.glo
%

\newglossaryentry{Beamforming}{name={Mikrofonarray},description={Beamforming ist ein Verfahren zur Positionsbestimmung von Quellen in Wellenfeldern (z. B. Schallfeldern). Entsprechende Vorrichtungen werden auch akustische Kamera, Mikrofonarray oder akustische Antenne genannt}}

\newglossaryentry{Kinect}{name={Kinect},description={Kinect ist eine Hardware zur Steuerung der Videospielkonsole Xbox 360, die seit Anfang November 2010 verkauft wird}}

\newglossaryentry{Bewegungsdetektion}{name={Bewegungsdetektion},description={Unter Bewegungsdetektion versteht man Methoden des Maschinellen Sehens, die auf die Erkennung von Fremdbewegung im Erfassungsbereich eines optischen Detektors (also dem Blickfeld der Maschine) abzielen}}

\newglossaryentry{Datenhandschuh}{name={Datenhandschuh},description={Der Datenhandschuh ist ein Eingabegerät in Form eines Handschuhs. Durch Bewegungen der Hand und Finger erfolgt eine Orientierung im virtuellen Raum},plural={Datenhandschuhe}}

\newglossaryentry{MMS}{name={Mensch-Maschine-Schnittstelle},description={Die Benutzerschnittstelle (nach Gesellschaft für Informatik, Fachbereich Mensch-Computer-Interaktion auch Benutzungsschnittstelle) ist die Stelle oder Handlung, mit der ein Mensch mit einer Maschine in Kontakt tritt},plural={Mensch-Maschine-Schnittstellen}}

\newglossaryentry{MCI}{name={Mensch-Computer-Interaktion},description={Die Mensch-Computer-Interaktion (englisch \emph{Human-Computer Interaction, HCI}) als Teilgebiet der Informatik beschäftigt sich mit der benutzergerechten Gestaltung von interaktiven Systemen und ihren Mensch-Maschine-Schnittstellen}}