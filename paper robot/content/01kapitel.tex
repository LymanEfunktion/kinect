\chapter{Einleitung}
\label{chap:Einleitung}

Seit jeher entwickeln wir Menschen Maschinen zur leichteren und effizienteren Verrichtung von Arbeiten. Um die Verrichtung von Arbeiten von Menschen effizient zu \"ubernehmen ist hier die denkbarste Form f\"ur einen Roboter eben auch die eines Menschen. Durch die Jahrhunderte haben Ingenieure und Wissenschaftler verschiedenste solcher Maschinen entwickelt. Die Entwicklung der ersten Automaten fand bereits in der Antike statt. Beispiele hierf\"uer sind die Wasserorgel von Heron von Alexandria, oder die fliegende Taube von Archytas von Tarent. Der arabische Erfinder al-Jazari, entwickelt zu Ende des zw\"olften Jahrhunderts verschiedene Automaten, wie zum Beispiel ein Tee-servierender Roboter. Im Mittelalter, im Jahr 1495, entwarf Leonardo Da Vinci einen Mechanischen Ritter, als humanoiden Androiden. Die Schweizer Br\"uder Droz und Droz bauen im Jahr 1774 drei humanoide Roboter, einen Zeichner, einen Schreiber und eine Klavierspielerin. 
\par\smallskip
Der Begriff Roboter wurde im Jahr 1920 vom tschechischen Schriftsteller Capek das erste mal in einem Schauspiel, Rossums Universal Robots\footnotemark[1], verwendet. Capek beschreibt hiermit Maschinenmenschen, welche Fronarbeit verrichten. Abgeleitet ist der Begriff vom tschechischen Wort für Fronarbeit, \glqq rabota \grqq. \newline
Dieses Motiv der Menschmaschine, wurde von Science Fiction Autoren aufgegriffen. Der ber\"uhmte Autor Isaac Asimov erw\"ahnte die Robotik als Studium der Roboter erstmals in seiner Kurzgeschichte Runaround\footnotemark[2], welche 1942 ver\"offentlicht wurde. Ebenso erfand Asimov in dieser Geschichte die 3 Robotergesetze:
\begin{itemize}
\item Ein Roboter darf keinen Menschen verletzen oder durch Untätigkeit zu Schaden kommen lassen.
\item Ein Roboter muss den Befehlen eines Menschen gehorchen, es sei denn, solche Befehle stehen im Widerspruch zum ersten Gesetz.
\item Ein Roboter muss seine eigene Existenz schützen, solange dieser Schutz nicht dem Ersten oder Zweiten Gesetz widerspricht.
\end{itemize}
\par\smallskip
Heute werden Roboter vor allem zur industriellen Fertigung eingesetzt. Sie ersetzen Menschen an Fließb\"andern, sowie weiteren Bereichen, wo die Arbeit durch einen Roboter effizienter und f\"ur den Menschen ungef\"ahrlicher durchgef\"uhrt werden kann. Die Logik dieser Roboter ist in der Regel fest einprogrammiert. Durch den immer breiteren Einsatz von Robotern, auch in zivilen Bereichen, wie beispielsweise der Pflege, wird es zunehmend wichtig auch hier neue Bedienkonzepte einzuf\"uhren. Da Roboter in diesen Bereichen zunehmend auch immer menschlicher dargestellt werden ist es hier wichtig auch dem Menschen nat\"urliche Bedienschnittstellen zu Verf\"ugung zu stellen. Die dem Menschen nat\"urlichsten Arten der Kommunikation sind Sprache und Gestik. Dieser Teil wurde von den Authoren bereits in einer ersten Arbeit behandelt \cite{bib:Ebner_Werling}. Diese Arbeit wird sich der Kombination der entwickelten Gesten- und Sprachbasierten Benutzerschnittstelle mit der Steuerung eines Roboters widmen.



\footnotetext[1]{Karel Capek. \href{http://ebooks.adelaide.edu.au/c/capek/karel/rur/}{Rossum Universal Robots von Capek, frei verf\"gbar bei der Universit\"at Adeleide}. ebooks.adelaide.edu.au. Abgerufen Mai 28, 2013}
\footnotetext[2]{Isaac Asimov. \href{http://www.rci.rutgers.edu/\textasciitilde cfs/472_html/Intro/NYT_Intro/History/Runaround.html}{Runaround von Asimov bei der RCI Rutgers}. rci.rutgers.edu. Abgerufen Mai 28, 2013}