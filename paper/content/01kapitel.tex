\chapter{Einleitung}
\label{chap:Einleitung}
Gestik im Sinne von kommunikativen Bewegungen stellt einen wesentlichen Teil der nonverbalen Kommunikation dar. Die Definition nach Kurtenbach und Hulteen (1990) besagt:
\begin{quote}
A gesture is a motion of the body that contains information. Waving goodbye is a gesture.
Pressing a key on a keyboard is not a gesture because the motion of a finger on its way to hitting a key on a keyboard is neither observed nor significant.
All that matters is which key was pressed.
\end{quote}
Als solches bildet eine Geste abstrakte Strukturen und bildhafte Vorstellungen unmittelbar ab. Sie ist somit eine K\"orperbewegung, 
die Informationen enth\"alt. Informationen, die herk\"ommliche Eingabeger\"ate, darunter Tastatur und Maus, nicht in dieser Form wiedergeben k\"onnen.
Eine Problematik, die sich besonders im Bereich der \gls{MCI} ergibt.
Das Ziel der \gls{MCI} ist es n\"amlich die Kommunikation so intuitiv und unmittelbar wie m\"oglich zu gestalten.
Genau an dieser Stelle setzen Gesten an, in dem sie eine bessere Schnittstelle zwischen Mensch und Maschine darstellen.
\newline
Fr\"uhe Ans\"atze, wie das \textbf{Theremin} - ein 1919 erfundenes elektornisches Musikinstrument,
das mit H\"anden ber\"uhrungslos durch Beeinflussung eines elektromagnetischen Feldes gesteuert wird und dadurch T\"one erzeugt -
oder \glspl{Datenhandschuh}, die erstmals 1983 von AT\& T Bell Labs entwickelt wurden zeigen, die gro\ss e Spannweite solcher \gls{MMS}