\pagestyle{empty}

\begin{abstract}
Durch den Fortschritt der Technik wird der Wunsch des Menschen sich bestimmte Arbeiten zu erleichtern immer gr\"o\ss er. Automaten, sogennante Roboter, werden eingesetzt um Menschen ihnen unliebsame, schwere oder auch gef\"ahrliche Aufgaben abzunehmen. Da Roboter mehr und mehr Eingang in den Alltag, durch immer weitere Anwendungsfelder, wie beispielsweise in der Pflege, finden, wird es wichtig sich mit der Interaktion von Mensch und Maschine zu befassen. \newline
In der ersten Arbeit wurde die Umsetzung der Schnittstelle zwischen Mensch und Maschine behandelt.\newline
Diese Arbeit besch\"aftigt sich mit der weiteren Entwicklung der Anwendung und im speziellen mit der Umsetzung der Schnittstelle zu einem mobilen Roboter, der durch das Gesten- und Sprachinterface ferngesteuert werden soll.
\par\smallskip
Die Autoren Ebner und Werling haben diese Arbeit gemeinsam verfasst, wobei die Kapitel jeweils von einem der Autoren geschreiben wurden. Auf Herrn Ebner entfallen Kapitel~\ref{chap:Aufgabenstellung},~\ref{chap:MenschComputerInteraktion},~\ref{chap:Konzept} und~\ref{chap:Implementierung} und auf Herrn Werling die Kapitel~\ref{chap:Einleitung},~\ref{chap:Technik},~\ref{chap:Roboterauswahl},~\ref{chap:Lego-Framework}, und ~\ref{chap:Ausblick}.
\end{abstract}
