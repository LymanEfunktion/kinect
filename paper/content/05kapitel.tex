\chapter{Konzeption}
\label{chap:Konzeption}

\section{Technische Vorgaben}
% Einf\"uhrende Worte

\subsection{Vorgaben durch Kinect}
% Entfernung zur Kinect \ldots Maximale Nutzerzahl \ldots, nur sicher nutzbar unter windwos \ldots

\subsection{Vorgaben durch jnect--Framework}
% Einschr\"ankung in Nutzung der KinectSchnittstelle \ldots entsprechende Entwurfsmuster n\"otig, nur eine Kinect \ldots

\subsection{Service--Architektur}
% Kurze Einf\"uhrung in OSGI und Begr\"undung

\subsection{Abh\"angigkeit zu Robotertechnik}
% Nutzbare Roboter einschr\"anken - argumentieren, Schnittstelle beschreiben

\section{Fachliche Vorgaben}
% Einf\"uhrende Worte, warum buxton, warum kategorisieren \ldots

\subsection{Vorgaben durch \gls{MCI}}
% was k\"onnen gesten, was soll geste f\"ur Schnittstelle k\"onnen etc.

\subsection{Klassifikation der Anwendung}
% Begr\"unding, warum semiotic \ldots weitere er\"orterungen nach buxton

\subsection{Analyse der Gesten- und Sprachsteuerung}
% Benötigt wird Kreisbewegung, Bewengung in Richtung mit Richtungsanpassung, und Haltezeichen - 3 + Sprache \ldots
% Modelle zur Erkennung in separatem Kapitel