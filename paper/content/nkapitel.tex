\chapter{Ausblick - weitere Arbeiten}
\label{chap:Ausblick}

Mit dieser Arbeit wurden die n\"oetigen Grundlagen geschaffen auf deren Basis in der nachfolgenden Arbeit die Ansteuerung 
eines mobilen Roboters erfolgen kann. Durch den Einsatz der in Kapitel~\ref{chap:Modelle} erarbeiteten Modelle wurde die 
M\"oglichkeit geschaffen Gesten auf Grund der Eingangsdaten der Kinect zu ermitteln. Die \glslink{HMM}{Hidden Markov Modells}
bilden auch die Grundlage der Erkennung von Worten aus dem Audiostrom. 

Welche Bedeutung der Einsatz von Gesten und Sprachen f\"ur ein User Interface hat wurden in den Kapiteln~\ref{chap:Gesten} und 
\ref{chap:Sprachbefehle} besprochen, sowie auch erste Entw\"urfe f\"ur Befehle, mit denen man einen Roboter steuern kann.
Festgestellt wurde, dass sich nicht jede Geste und jeder Sprachbefehl f\"ur eine Schnittstelle zwischen Mensch und Maschine eignen
und das vor allem auch die Nat\"urlichkeit f\"ur den Anwender wichtig ist.

Auf Basis des in Kapitel~\ref{chap:Kinect} gew\"ahlten Frameworks zur Ansteuerung der Kinect, jnect, wurde eine, der in 
Kapitel~\ref{chap:Konzeption} vorgestellten Konzeption folgende, Anwendung zur Erfassung von Gesten und Sprachbefehlen entwickelt. 
Diese bietet eine Schnittstelle zur leichten Integration der, in der zweiten Arbeit folgenden, Roboterschnittstelle.

Im zweiten Teil der Arbeit werden die bisher erdachten Gesten und Sprachbefehle noch weiter spezifiziert und auch mit Hilfe
eines Roboters umgesetzt werden. In diesem Rahmen wird auch die Anwendung um die Roboterschnittstelle erweitert, sowie die
bestehende Implementierung nochmals betrachtet und überarbeitet. Zudem werden die notwendigen Konzepte zur Interaktion 
zwischen Mensch und Roboter, sowie die Grundlagen zur Steuerung des gew\"ahlten Roboters beleuchtet werden. % The End